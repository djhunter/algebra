% Options for packages loaded elsewhere
\PassOptionsToPackage{unicode}{hyperref}
\PassOptionsToPackage{hyphens}{url}
%
\documentclass[
  twoside]{article}
\usepackage{amsmath,amssymb}
\usepackage{iftex}
\ifPDFTeX
  \usepackage[T1]{fontenc}
  \usepackage[utf8]{inputenc}
  \usepackage{textcomp} % provide euro and other symbols
\else % if luatex or xetex
  \usepackage{unicode-math} % this also loads fontspec
  \defaultfontfeatures{Scale=MatchLowercase}
  \defaultfontfeatures[\rmfamily]{Ligatures=TeX,Scale=1}
\fi
\usepackage{lmodern}
\ifPDFTeX\else
  % xetex/luatex font selection
\fi
% Use upquote if available, for straight quotes in verbatim environments
\IfFileExists{upquote.sty}{\usepackage{upquote}}{}
\IfFileExists{microtype.sty}{% use microtype if available
  \usepackage[]{microtype}
  \UseMicrotypeSet[protrusion]{basicmath} % disable protrusion for tt fonts
}{}
\makeatletter
\@ifundefined{KOMAClassName}{% if non-KOMA class
  \IfFileExists{parskip.sty}{%
    \usepackage{parskip}
  }{% else
    \setlength{\parindent}{0pt}
    \setlength{\parskip}{6pt plus 2pt minus 1pt}}
}{% if KOMA class
  \KOMAoptions{parskip=half}}
\makeatother
\usepackage{xcolor}
\usepackage[left=1.0in, right=1.0in, top=0.8in, bottom=0.4in]{geometry}
\usepackage{graphicx}
\makeatletter
\def\maxwidth{\ifdim\Gin@nat@width>\linewidth\linewidth\else\Gin@nat@width\fi}
\def\maxheight{\ifdim\Gin@nat@height>\textheight\textheight\else\Gin@nat@height\fi}
\makeatother
% Scale images if necessary, so that they will not overflow the page
% margins by default, and it is still possible to overwrite the defaults
% using explicit options in \includegraphics[width, height, ...]{}
\setkeys{Gin}{width=\maxwidth,height=\maxheight,keepaspectratio}
% Set default figure placement to htbp
\makeatletter
\def\fps@figure{htbp}
\makeatother
\setlength{\emergencystretch}{3em} % prevent overfull lines
\providecommand{\tightlist}{%
  \setlength{\itemsep}{0pt}\setlength{\parskip}{0pt}}
\setcounter{secnumdepth}{-\maxdimen} % remove section numbering
\usepackage[charter]{mathdesign}
\usepackage{fancyhdr}
\pagestyle{fancy}
\lhead{MA-110, Hunter, Fall 2023}
\rhead{Syllabus, Page \thepage}
\cfoot{}
\setlength{\headsep}{0.2in}

\usepackage{enumitem}

\setenumerate{leftmargin=*}

\newcommand{\vect}[1]{\mathbf{#1}} % vectors in bold face

%\newcommand{\vect}[1]{\mathbf{#1}\,} % vectors in bold face (need thin
                                     % space after)
%\newcommand{\vect}[1]{\vec{#1}} %vectors as arrows

\providecommand{\norm}[1]{\left\lvert#1\right\rvert} % norms as single lines
%\providecommand{\norm}[1]{\left\lVert#1\right\rVert} % norms as double lines

% bold or blackboard bold!?
\newcommand{\NN}{\mathbb{N}}
%\newcommand{\RR}{\mathbf{R}}
\newcommand{\RR}{\mathbb{R}}

% some useful abbreviations
\newcommand{\vx}{\vect{x}}
\newcommand{\vy}{\vect{y}}
\newcommand{\vz}{\vect{z}}
\newcommand{\vu}{\vect{u}}
\newcommand{\vv}{\vect{v}}
\newcommand{\vw}{\vect{w}}
\newcommand{\va}{\vect{a}}
\newcommand{\vb}{\vect{b}}
\newcommand{\vc}{\vect{c}}
\newcommand{\ve}{\vect{e}}
\newcommand{\vf}{\vect{f}}
\newcommand{\vF}{\vect{F}}
\newcommand{\vg}{\vect{g}}
\newcommand{\vh}{\vect{h}}
\newcommand{\vl}{\vect{l}}
\newcommand{\vm}{\vect{m}}
\newcommand{\vn}{\vect{n}}
\newcommand{\vp}{\vect{p}}
\newcommand{\vr}{\vect{r}}
\newcommand{\vs}{\vect{s}}
\newcommand{\vi}{\vect{i}}
\newcommand{\vj}{\vect{j}}
\newcommand{\vk}{\vect{k}}
\newcommand{\vzero}{\vect{0}}
% lower-case greek letters are handled differently: poor man's bold macro
%\newcommand{\vphi}{\pmb{\phi}}

\newcommand{\malename}{Westley} % recurring male name
\newcommand{\femalename}{Buttercup} % recurring female name
\usepackage{booktabs}
\usepackage{longtable}
\usepackage{array}
\usepackage{multirow}
\usepackage{wrapfig}
\usepackage{float}
\usepackage{colortbl}
\usepackage{pdflscape}
\usepackage{tabu}
\usepackage{threeparttable}
\usepackage{threeparttablex}
\usepackage[normalem]{ulem}
\usepackage{makecell}
\usepackage{xcolor}
\ifLuaTeX
  \usepackage{selnolig}  % disable illegal ligatures
\fi
\IfFileExists{bookmark.sty}{\usepackage{bookmark}}{\usepackage{hyperref}}
\IfFileExists{xurl.sty}{\usepackage{xurl}}{} % add URL line breaks if available
\urlstyle{same}
\hypersetup{
  hidelinks,
  pdfcreator={LaTeX via pandoc}}

\author{}
\date{\vspace{-2.5em}}

\begin{document}

\hypertarget{abstract-algebra-ma-110-westmont-college-fall-2023}{%
\section{Abstract Algebra (MA-110) Westmont College, Fall
2023}\label{abstract-algebra-ma-110-westmont-college-fall-2023}}

\hypertarget{what-is-this-course-about}{%
\subsection{What is this course
about?}\label{what-is-this-course-about}}

Have you ever wondered how modern mathematics came to be? To what extent
are the definitions and theorems that we find in the canonical textbooks
the result of choices made by historical figures? Is there something
intrinsic to the subject that forces modern mathematical theory down the
path that it takes? Put another way, would Earthling mathematicians
recognize the work of a community of mathematicians from a distant
galaxy?

We can glean some insight into these questions by proceeding
axiomatically: make as few assumptions and definitions as possible, and
see where logical deduction leads. The material in this course (and the
sequel, MA-111) represents the consequences of some very low-level
mathematical foundations: logic, sets, and the integers. These
consequences have applications throughout pure and applied mathematics,
and are part of the \textit{lingua franca} of the discipline.

Historically, much of this subject was motivated by the search for
solutions to polynomial equations: are there analogs to the quadratic
formula for polynomials of degree greater than 2? It is easy to believe
that our hypothetical space-alien mathematical colleagues would also be
interested in this question. Throughout this semester and the next
(should you persevere), you will be guided through a series of deductive
discoveries along the path that modern mathematicians have blazed.
Whether this path is the only reasonable one will be yours to judge once
you have completed the journey.

\hypertarget{cool.-ok.-but-what-does-the-catalog-say-the-course-is-about}{%
\subsection{Cool. OK. But what does the catalog say the course is
about?}\label{cool.-ok.-but-what-does-the-catalog-say-the-course-is-about}}

``(Four credit hours) Prerequisite: MA 020. Abstract algebraic patterns
pervade modern mathematics. Given simple definitions of groups, rings,
and fields, this course develops the intricate theory of these objects,
including permutation groups, subgroups, quotient groups and rings,
isomorphisms, and extension fields.''

\hypertarget{how-is-the-course-structured}{%
\subsection{How is the course
structured?}\label{how-is-the-course-structured}}

We will work through a series of problems that, along with the sequel,
give a comprehensive coverage of two semesters of advanced undergraduate
or beginning graduate algebra. Ideally, you will supply the answers.
Your written accounts of your inquiry and our class discussions will
produce a complete modern algebra textbook of your own construction.

This course is meant to prepare you for your future studies and work.
You will inevitably encounter mathematical concepts in situations where
nobody is there to explain things to you. I want you to develop the
necessary skills of inquiry to prepare you for these encounters.
Therefore, I will often be expecting you to find answers to problems
yourself. There will be times when I intentionally leave things
unexplained for you to figure out. This is part of an educational
strategy designed to make you a better thinker.

Unless you are told otherwise, \textbf{do not use outside resources}
(e.g., Internet, books) when you work on the assignments. I would rather
see a half-right solution that you have constructed yourself than a
perfect solution that somebody else constructed.

This course presents an opportunity for you to grow in your ability to
work by yourself. I strongly recommend that you spend time alone with
the assigned problems before working with others. If you do work with
others on an assignment, please include a note on your assignment
indicating the names of those you worked with. And remember that
\textbf{the work you hand in should represent your own understanding.}

There will typically be three phases to the inquiry process, all focused
on the sequence of problems from the text.

\begin{description}
          \item[Prework:] On most days, there will be a written assignment due the night before class. My expectation is that these prework assignments will be hand written; I encourage you to find an efficient notebook system that allows you to intersperse hand-outs and your own written work. A three-ring binder works well for this purpose.
          
Use a scanner app (e.g., Genius Scan, Adobe Scan) to make a PDF of your prework and submit it on Canvas. If your prework spans multiple pages, please combine all pages into a single PDF file.
          
In general, you can receive full credit for less-than-perfect prework assignments, as long as you make significant progress on each problem.  Making progress can include writing down the first and last lines of the proof, and formulating some questions about what you didn't understand.
          \item[Class Discussion:] The bulk of our class time will be spent discussing the assigned problems that you have attempted in the prework phase. On most days, you will be assigned one or more of the problems to present at the board. First,  students write their solutions (or partial solutions) on the board. Once everyone has finished, we will discuss all the problems together in whole-class discussion. During these discussions you should focus on correcting and completing your prework assignments. Since there is not a traditional textbook for this class, you have the responsibility of recording your own complete solutions for every problem.
          \item[Final Drafts:]  Each week, a selection of problems will be assigned for you to typeset a final draft of solutions. These final drafts will be graded on logic, exposition, and formatting. You should add your graded and corrected final drafts to your binder, which will grow into a textbook of your own construction.
      \end{description}

\hypertarget{is-there-a-textbook}{%
\subsection{Is there a textbook?}\label{is-there-a-textbook}}

The skeleton of the text for this course is a revised and expanded
version of \textit{Introductory Abstract Algebra}, by Margaret L.
Morrow, \textit{Journal of Inquiry Based Learning in Mathematics}
\textbf{26}, (2012). Morrow's notes provide a gentle, inquiry-based
introduction to group theory. I have added several chapters to these
notes covering rings and fields. Throughout the semester, you will be
adding pages of notes, prework, and final drafts to your binder, which
will evolve into a complete text that you have taken part in writing.

\hypertarget{how-are-grades-determined}{%
\subsection{How are grades
determined?}\label{how-are-grades-determined}}

You will receive regular grades on your written prework and typeset
final drafts. You will find grades and feedback on Canvas. In addition,
there will be four traditional exams. Grades are weighted as follows.

\begin{tabular}[t]{ll}
\toprule
Written Prework Assignments & 15\%\\
Typeset Final Drafts & 20\%\\
Exams (3) & 15\% each\\
Final Exam & 20\%\\
\bottomrule
\end{tabular}

\hypertarget{what-other-policies-should-students-be-aware-of}{%
\subsection{What other policies should students be aware
of?}\label{what-other-policies-should-students-be-aware-of}}

If you miss a significant number of classes, you will almost definitely
do poorly in this class. If you miss more than six classes without a
valid excuse, I reserve the right to terminate you from the course with
a failing grade. Work missed (including tests) without a valid excuse
will receive a zero.

I expect you to check your email on a regular basis. If you use a
non-Westmont email account, please forward your Westmont email to your
preferred account. I'll send out notices on Canvas, so make sure you
receive Canvas notifications in your email.

Learning communities function best when students have academic
integrity. Cheating is primarily an offense against your classmates
because it undermines our learning community. Therefore, dishonesty of
any kind may result in loss of credit for the work involved and the
filing of a report with the Provost's Office. Major or repeated
infractions may result in dismissal from the course with a failing
grade. Be familiar with the College's plagiarism policy, found at
\url{https://www.westmont.edu/office-provost/academic-program/academic-integrity-policy}.

In particular, providing someone with an electronic copy of your work is
a breach of the academic integrity policy. Do not email, post online, or
otherwise disseminate any of the work that you do in this class. If you
keep your work on a repository, make sure it is private. You may work
with others on the assignments, but make sure that you write or type up
your own answers yourself. You are on your honor that the work you hand
in represents your own understanding.

\clearpage

\hypertarget{other-information}{%
\subsection{Other Information}\label{other-information}}

\begin{description} 

\item[Professor:] David J. Hunter, Ph.D.
  (\verb!dhunter@westmont.edu!). Student hours are from 1:00--3:30pm on Tuesdays and Thursdays in Winter Hall 303.

 \item[Tentative Schedule:] We will try to conform to the following
  schedule, although it is subject to revision at the instructor's
  discretion.
  \medskip

  { \renewcommand{\arraystretch}{1.1}
\begin{tabular}{||l|l||}
\hline
\textbf{Section} & \textbf{Topic} \\
\hline\hline
 1 & Introduction to Groups \\ \hline
 2 & New Groups from Old \\ \hline
 2.1 & Direct Products of Groups \\ \hline
 2.2 & Subgroups \\ \hline
 2.3 & Cyclic Subgroups \\ \hline
 3 & Homomorphisms and Isomorphisms \\ \hline
 3.1 & Definitions \\ \hline
 3.2 & Preservation of Structure \\ \hline
 3.3 & Kernel and Image \\ \hline
  & \textbf{First Exam} \\ \hline
 4 & Cosets: First Ideas and Applications \\ \hline
 4.1 & Definition and Examples \\ \hline
 4.2 & Cosets and Partitions \\ \hline
 4.3 & Cosets and Group Order \\ \hline
 5 & Normal Subgroups and Quotient Groups \\ \hline
 5.1 & Normal Subgroups \\ \hline
 5.2 & Quotient Groups \\ \hline
 5.3 & Isomorphism Theorems \\ \hline
 6 & Rings and Ideals \\ \hline
 6.1 & Rings \\ \hline
 6.2 & Homomorphisms and Kernels \\ \hline
 6.3 & Ideals \\ \hline
  & \textbf{Second Exam} \\ \hline
 7 & Quotient Rings \\ \hline
 7.1 & Cosets \\ \hline
 7.2 & Constructing Quotient Rings \\ \hline
 7.3 & Isomorphisms \\ \hline
 8 & Divisibility \\ \hline
 8.1 & Integral Domains and Fields \\ \hline
 8.2 & The Division Algorithm \\ \hline
 8.3 & Greatest Common Divisor \\ \hline
 8.4 & The Euclidean Algorithm \\ \hline
  & \textbf{Third Exam} \\ \hline
 9 & Prime Ideals and Maximal Ideals \\ \hline
 9.1 & Roots of Polynomials \\ \hline
 9.2 & Prime Ideals \\ \hline
 9.3 & Maximal Ideals \\ \hline
 9.4 & Field Extensions \\ \hline
  & \textbf{Final Exam} \\ \hline
\end{tabular}
}

\clearpage

\item[General Education Writing Intensive Course:] This course is
  classified as ``writing intensive'' for the purposes of Westmont's general
  education requirements.  Therefore we will spend a significant amount of time
  learning how to write mathematically.  Your written prework assignments will contain many
  problems where you are asked to ``prove'' something, and I will grade these
  proofs taking both correctness and exposition into account.  A selection of problems will be assigned for you to typeset carefully and submit a final draft, doing revisions as necessary. I expect you all to make progress in
  your ability to write mathematically.

\item[Accommodations for Students with Disabilities:] Students who have been diagnosed with a disability (learning, physical or psychological) are strongly encouraged to contact the Disability Services office as early as possible to discuss appropriate accommodations for this course. Formal accommodations will only be granted for students whose disabilities have been verified by the Disability Services office.  These accommodations may be necessary to ensure your equal access to this course.  Please contact the Office of Disability Services (\href{mailto:ods@westmont.edu}{\tt ods@westmont.edu}) or visit \url{https://www.westmont.edu/disability-services} for more information.

\item[Program and Institutional Learning Outcomes:] The
         mathematics department at Westmont College has formulated the
         following learning outcomes for all of its classes. (PLO's)
\begin{enumerate}[noitemsep]
\item Core Knowledge: Students will demonstrate knowledge of the
                  main concepts, skills, and facts of the discipline of
                  mathematics.
\item Communication: Students will be able to communicate mathematical ideas
     following the standard conventions of writing or speaking in the
     discipline.
\item Creativity: Students will demonstrate the ability to formulate and make
     progress toward solving non-routine problems.
\item Christian Connection: Students will incorporate their mathematical skills
     and knowledge into their thinking about their vocations as followers of
     Christ.
         \end{enumerate}
         In addition, the faculty of Westmont College have established common
         learning outcomes for all courses at the institution
         (ILO's). These outcomes are summarized as follows:
(1) Christian Understanding, Practices, and Affections,
(2) Global Awareness and Diversity,
(3) Critical Thinking,
(4) Quantitative Literacy,
(5) Written Communication,
(6) Oral Communication, and
(7) Information Literacy.

\item[Course Learning Outcomes:] The above outcomes are reflected in the
     particular learning outcomes for this course.
     After taking this course, you should be able
     to:
    \begin{itemize}
        \item Demonstrate understanding of modern abstract algebra.
             (PLO 1, ILOs 3,4)
        \item Write mathematical arguments according to the
             standards of the discipline. (PLO 2,
              ILOs 3,5)
        \item Construct solutions to novel problems,
               demonstrating perseverance in the face of open-ended or
               partially-defined contexts. (PLO 3, ILO 3)
        \item Consider the theological implications of the subject matter. (PLO 4, ILO 1)
    \end{itemize}
These outcomes will be assessed by written prework assignments, typeset final drafts, and written exams, as described above.

\end{description}

\end{document}
